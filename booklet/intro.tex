\section{Introduction}
In Robotics one of the most interesting task is the manipulation of objects. In general
it can be distinguished between the task of restraining objects, called grasping,
and the task of manipulating objects with \emph{fingers} called dexterous manipulation.
\par
Among the hand-like end-effectors equipped with fingers proposed in the literature
there is the Pisa/IIT SoftHand. It is simple because driven by one motor only
but at the same time robust because it adapts itself to the shape of grasped objects.
\par
Altough very versatile the simplicity of the hand make the grasp of thin objects,
such as a credit card or a sheet of paper, a very hard task
because such objects do not provide enough contact constraints on the hand
for the grasp to take place.
\par
One possible solution to this problem is to use environmental constraints, such as
the surface of a table, to drag the object until it reaches the edge of the table and
sticks out of it. Then a standard grasp can be performed as usual. However in order to make
this solution workable the problem of making the dragging phase \emph{safe} must be faced.
In fact by using a \emph{pure position} control strategy the \emph{uncontrolled} contact forces and torques
arising between the hand and the object and between the object and the surface of the table
could damage the object or the hand itself.
\par
In this project an \emph{hybrid force position} controller was implemented and tested on the
scenario described above to show that a force feedback solution allows to accomplish the task
without the issue of damaging the dragged object or the end-effector.
\par
The implementation was done using the ROS Control architecture and the robot used
in the experimental setup was a KUKA LWR $4+$ manipulator.

\subsection{Contents of the report}

\newpage
