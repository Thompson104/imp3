\section{Inverse Dynamics Inner Loop}
In the previous section the hybrid force position control scheme was described
with the assumption that an inverse dynamics inner loop was available. In this
section the inverse dynamics controller is described and the issue of controlling
a kinematically redudant manipulator is discussed.

\subsection{Design of the inverse dynamics controller}
The development of the controller starts from the dynamic model of the 7-joint
manipulator
\begin{equation}\label{eq:joint_space_dyn}
  B(\vec{q}) \ddot{\vec{q}} + C(\vec{q}, \dot{\vec{q}}) \dot{\vec{q}} + \cancel{\vec{G}(\vec{q})} = \vec{\tau}
  -\prescript{b}{}{J}^{T}_{F} \prescript{b}{}{\vec{w}}_{F}
\end{equation}

where $\vec{w}_{F}$ is the wrench exerted by the end-effector on the environment with reference point in the centre
of the wrist of the robot ($F$) and $J_{F}$ is the geometric Jacobian of the robot.
The gravity term is canceled because it is handled internally by the robot itself.
\par
The first step towards the exact dynamics inversion is to cancel the Coriolis term
$C \dot{\vec{q}}$ and the term containing the wrench $\vec{w}_{F}$ which is supposed to
be available as a feedback signal. The resulting control torque is then
\[
\vec{\tau} = C \dot{\vec{q}} + {}^{b}J^{T}_{F} ({}^b\vec{\gamma} + {}^b\vec{w}_{F})
\]
where an additional control wrench $\vec{\gamma}$ is introduced.
Substituing the control torque and solving for the joints accelerations $\vec{\ddot{q}}$
leads to
\begin{equation}\label{eq:qddot}
  \ddot{\vec{q}} = B^{-1} ({}^{b}J^{T}_{F}) {}^b\vec{\gamma}
\end{equation}
\par
The link between the joint space description and the operational space description
is given by the following formula
\begin{equation}
  \label{eq:ws_a}
  {}^{ws} \ddot{\vec{x}} = {}^{ws} J_{A,E} \ddot{\vec{q}} + {}^{ws} \dot{J_{A,E}} \dot{\vec{q}}
\end{equation}
where ${}^{ws} \vec{x}$ is the previously introduced (cite) state in operational space expressed in workspace
frame and $J_{A,E}$ is the analytical Jacobian of the robot with reference point in the palm of the hand.
The analytical Jacobian, which takes into account the conversion between the angular velocity $\vec{\omega}$
and the vector of angular rates $\vec{\dot{\Phi}}$, can be written as
\[
\prescript{ws}{}{J}_{A,E} = 
\begin{bmatrix}
  I & 0 \\
  0 & T^{-1}(\vec{\Phi})
\end{bmatrix}
\prescript{ws}{}{J}_{E} \quad {}^{ws}\vec{\omega} = T(\vec{\Phi}) \vec{\dot{\Phi}}
\]
with $J_E$ the geometric Jacobian with reference point int the palm of the hand. It should be noted
that the inverse of the matrix $T$ exists if and only if the sine of the angle $\theta$ (part of the state
$\vec{x}$) is not zero. The derivative of the analytical Jacobian evaluates to
\[
\prescript{ws}{}{\dot{J}}_{A,E} = 
\begin{bmatrix}
  0 & 0 \\
  0 & -T^{-1} \dot{T} T^{-1}
\end{bmatrix}
\prescript{ws}{}{J}_{E} + 
\begin{bmatrix}
  I & 0 \\
  0 & T^{-1}
\end{bmatrix}
\prescript{ws}{}{\dot{J}}_{E}
\]
where the derivative of the Jacobian $\dot{J}_{E}$ can be evaluated numerically
using software libraries like KDL (Kinematics and Dynamics Library).
\par
By substituing equation (\ref{eq:qddot}) in equation (\ref{eq:ws_a}) one obtains
\[
\prescript{ws}{}{\ddot{\vec{x}}} = \underbrace{{}^{ws} J_{A,E} \B^{-1} ({}^b J_{F}^T)}_{\Lambda_A^{-1}} {}^b \vec{\gamma} + {}^{ws} \ddot{J_{A,E}} \dot{\vec{q}}
\]
where the kinetic psuedo-kinetic energy matrix $\Lambda_{A}$ was introduced. Multiplying both sides of the equation
by $Lambda_{A}$ leads to
\[
\Lambda_A {}^{ws} \ddot{\vec{x}} =  {}^b \vec{\gamma} +  \Lambda_A  {}^{ws} \dot{J_{A,E}} \dot{\vec{q}}
\]
where the control wrench $\gamma$ can now be easily chosen as
\[\label{eq:gamma}
\vec{\gamma} = \Lambda_A \vec{a} - \Lambda_A {}^{ws} \dot{J_{A,E}} \dot{\vec{q}}
\]
resulting in
\[
\prescript{ws}{}{\ddot{\vec{x}}} = \vec{a}
\]
where $\vec{a}$ represents the outer control loop whose design was described in the previous
section. Taking into account the command wrench $\vec{\gamma}$ the final commanded torque is
\begin{equation}\label{eq:torque_wo_null}
  \vec{\tau} = C \dot{\vec{q}} + {}^{b}J^{T}_{F} ( B_A \vec{a} - B_A {}^{ws} \dot{J_{A,E}} \dot{\vec{q}} + {}^b\vec{w}_{F})
\end{equation}

\subsection{Control of redundant manipulators}
The control law presented in the previous subsection does not take into account
the fact that the controlled manipulator is kinematically redundant. In fact it has
7 joints while the commanded state in the operational space is 6-dimensional.
A consequence of this fact is that [Khatib] the configuration of the robot, which describes the
end-effector position and orientation, does not constitute a generalized coordinate
system for the whole redundant manipulator and the dynamic behavior of the entire redundant
system cannot be represented by a dynamic model in coordinates only of the end-effector
configuration. Indeed it turns out that the command torques specified in equation (\ref{eq:torque_wo_null}),
altough they specify univocally the behavior of the end-effector, they are none other than one of the
possible torques producing a given operational command wrench $\vec{\gamma}$ on the end-effector, i.e,
the expression
\begin{equation}\label{eq:torques_gamma}
  \vec{\tau} = J^{T} \vec{\gamma}
\end{equation}
represents just one of these torques. Clearly different torques lead to completely different behaviors
of the additional degree of freedom of a kinematically redundant manipulator, some of which may be undesired
or even dangerous. For this reason a complete characterization of the torques producing a given command wrench
at the end-effector is necessary to avoid this issue.
\par
Independently of the specific choice for the operational space description of the robot
the second derivative with respect to time of the state can be written in the form
\begin{equation}\label{eq:xddot}
  \vec{\ddot{x}} = J(\vec{q}) \vec{\ddot{q}} + \vec{h}(\vec{q},\vec{\dot{q}})
\end{equation}
for some Jacobian matrix $J$ and some function of the joints state and velocity $\vec{h}$.
By combining equations (\ref{eq:joint_space_dyn}), in the case of zero contact forces,
(\ref{eq:torques_gamma}) and (\ref{eq:xddot}) the equations of motion of the end-effector
are obtained
\begin{equation}\label{eq:operational_space_dyn}
  \Lambda \vec{\ddot{x}} + \vec{\mu}(\vec{q}, \vec{\dot{q}}) + \vec{p} = \vec{\gamma}
\end{equation}
where
\[
\Lambda = (J B^{-1} J^{T})^{-1}
\]
\[
\vec{\mu} = \bar{J}^{T} C \vec{\dot{q}} - \Lambda \vec{h}
\]
\[
\vec{p} = \bar{J}^T \vec{G}
\]
are the already seen pseudo-kinetic energy matrix, the centrifugal and
Coriolis forces acting on the end-effector and the gravity force acting on the end-effector
respectively.
The matrix $\bar{J} = B^{-1} J^{T} \Lambda$ is a generalized inverse of the Jacobian matrix
which solves the problem of finding the joints velocities that produce a desired twist
while minimizing the manipulator's instantaneous kinetic energy.
\par
Interestingly equation (\ref{eq:operational_space_dyn}) can also be written in the form
\[
\bar{J}^{T} (B \vec{\ddot{q}} + C \vec{\dot{q}} + \vec{G}) = \vec{\gamma}
\]
Since the joint space dynamics term that is multiplied by $\bar{J}^{T}$ is equal to the
commanded torque $\vec{\tau}$ one finds that
\begin{equation}\label{eq:gamma_torques}
  \vec{\gamma} = \bar{J}^{T} \vec{\tau}
\end{equation}
which is an important result since it allows to design a command torque $\vec{\tau}$ of the form
\begin{equation}\label{eq:torque_w_null}
  \vec{\tau} = J^{T} \vec{\gamma} + (I_7 - J^{T}\bar{J}^{T}) \vec{\gamma}_{0}
\end{equation}
where the command wrench $\vec{\gamma}_{0}$ is projected in the null space of the
generalized inverse $\bar{J}^{T}$ hence, as results by direct substitution of equation
(\ref{eq:torque_w_null}) in equation (\ref{eq:gamma_torques}), it corresponds
to a \emph{null} operational wrench command. Clearly the term $\vec{\gamma}_{0}$ can
be used to avoid undesired motion of the additional degree of freedom of the manipulator
without interfering with the desired motion of the end-effector.
\par
It should be noted that the expression in equation (\ref{eq:torque_w_null}) is very general, i.e.,
every commanded torque $\vec{\tau}$ can always be expressed in that form.
Also due to its filtering capabilities the matrix $\bar{J}$ is called \emph{dynamically consistent generalized inverse} 
of $J$ in the sense that the projector in the null of $\bar{J}^T$ is able 
to filter out commanded wrenches in a way that is consistent with the dynamics of both the
whole manipulator and the end-effector.
\par
Taking into account the development presented in this section the commanded torques $\vec{\tau}$
in equation (\ref{eq:torque_wo_null}) are updated to
\begin{equation}\label{eq:torqu_w_null_1}
  \vec{\tau} = C \dot{\vec{q}} + {}^{b}J^{T}_{F} ( B_A \vec{a} - B_A {}^{ws} \dot{J_{A,E}} \dot{\vec{q}} + {}^b\vec{w}_{F}) +
  (I_7 - ({}^{b}J^{T}_{F}) ({}^{b} \bar{J}^{T}_{F})) \vec{\gamma}_{0}
\end{equation}
where
\[
\prescript{b}{}{\bar{J}}_{F} = B^{-1} ({}^{b} J^{T}_{F}) ({}^{b} \Lambda_{F})
\]
and
\[
\prescript{b}{}{\Lambda}_{F} = ({}^{b} J_{F} B^{-1} ({}^{b} J^{T}_{F}))^{-1}
\]

\subsubsection{Design of the null operational wrench command}
Design of the control law assigned to the term $\vec{\gamma}_{0}$ was
guided by the simulation of the task described in section (?). Extensive simulations
revealed that the uncontrolled internal motions of the robot, while not affecting the desired
attitude of the end-effector, caused the $5$-th and $7$-th links to rotate cooperatively and reach
their limits soon. Another issue with internal motions is that they could cause, in some situations,
collisions between the $4$-th link and the table which is part of the workspace of the robot.
\par
In order to handle the issues described the control $\vec{\gamma}_0$ was choosen as
\[
\vec{\gamma}_{0} = J_{im} ^{T} (K_p (\vec{x}_{im,des} - \vec{x}_{im} ) - K_d \vec{\dot{x}}_{im})
\]
where \emph{im} stands for ``internal motion'', the state $\vec{x}_{im}$
\[
\vec{x}_{im} =
\begin{bmatrix}
  {}^{b} x_{l4} & {}^{b} y_{l4} & {}^{b} z_{l4} & \psi_{l5} & \theta_{l5} & \phi_{l5}
\end{bmatrix}^{T} =
\begin{bmatrix}
  \vec{p}^{T}_{l4} & \vec{\Phi}^{T}_{l5}
\end{bmatrix}^{T}
\]
describes the position, with respect to the base of the robot, of the $4$-th link
and the attitude of the $5$-th link using a ZYZ parametrization and the Jacobian
$J_{im}$ is such that
\[
\begin{bmatrix}
  \vec{\dot{p}}_{l4} \\
  \vec{\dot{\Phi}}_{l5} \\
\end{bmatrix} =
J_{im} \vec{\dot{q}}
\]
\par
The gains $K_p$ and $K_d$ and the desired state $\vec{x}_{im,des}$ were chosen as
\[
K_p =
\begin{bmatrix}
  0 & 0 & 0 & 0 & 0 & 0\\
  0 & 0 & 0 & 0 & 0 & 0\\
  0 & 0 & k_{p,z}^{im} & 0 & 0 & 0\\
  0 & 0 & 0 & 0 & 0 & 0\\
  0 & 0 & 0 & 0 & 0 & 0\\
  0 & 0 & 0 & 0 & 0 & k_{p, att}^{im}\\
\end{bmatrix}
\quad
K_d = k_{d}^{im} I_{6}
\]
\[
\vec{p}_{l4} =
\begin{bmatrix}
  * & * & {}^{b} z_{ee} + off_z
\end{bmatrix}^{T}
\quad
\vec{\Phi}_{l5} =
\begin{bmatrix}
  * & * & 0\\
\end{bmatrix}^{T}
\]
where ${}^{b} z_{ee}$ is the altitude of the end-effector with respect to the base of the robot
and $k_{p,z}^{im}$, $k_{att}^{im}$ and $k_{d}^{in}$ are suitable gains.
Such a choice allows to fix the orientation of the $5$-th link to a value that is comfortable
during the execution of task described in (??) and to regulate the altitude between the $4$-th link and
the end-effector so as to avoid collisions between the elbow of the robot and the workspace.
