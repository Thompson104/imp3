\begin{frame}{Issues with internal motions}
  In the case of the task described the Kuka LWR $4+$ is a kinetically redundant manipulator.
  \par
  Extensive simulations revealed
  that the uncontrolled internal motions of the robot, while not affecting the
  desired attitude of the hand, cause the $5$-th and $7$-th links to rotate
  cooperatively and reach their limits soon. Another issue with internal motions
  is that they could cause, in some situations, collisions between the $4$-th
  link and the table which is part of the workspace of the robot.
  \par
  These issues are solved use a \emph{dynamically consistent} generalized inverse of the Jacobian [Khatib, 1987]
\end{frame}
  
\begin{frame}{Control of internal motions [Khatib, 1987]}
  The standard operational space dynamics can be written by substituting
  (\ref{eq:xddot}) in (\ref{eq:dynamic_joint}) resulting in (\ref{eq:operational_space_dyn})
  \begin{columns}
    \begin{column}{0.7\columnwidth}
      \begin{equation}\label{eq:dynamic_joint}
      B(\vec{q}) \ddot{\vec{q}} + C(\vec{q}, \dot{\vec{q}}) \dot{\vec{q}} + \vec{G}(\vec{q}) = \vec{\tau} = J^{T} \vec{\gamma}
      \end{equation}
      \begin{equation}\label{eq:xddot}
        \vec{\ddot{x}} = J(\vec{q}) \vec{\ddot{q}} + \vec{h}(\vec{q},\vec{\dot{q}})
      \end{equation}
      \begin{equation}\label{eq:operational_space_dyn}
        \Lambda \vec{\ddot{x}} + \vec{\mu}(\vec{q}, \vec{\dot{q}}) + \vec{p} = \vec{\gamma}
      \end{equation}
    \end{column}
    \begin{column}{0.4\columnwidth}
      \[
      \begin{split}
        &\Lambda = (J B^{-1} J^{T})^{-1}\\
        &\vec{\mu} = \bar{J}^{T} C \vec{\dot{q}} - \Lambda \vec{h}\\
        &\vec{p} = \bar{J}^T \vec{G}\\
        &\bar{J} = B^{-1} J^{T} \Lambda
      \end{split}
      \]
    \end{column}
  \end{columns}
  %% The matrix 
  %% which solves the problem of finding the joints velocities that produce a desired twist
  %% while minimizing the manipulator's instantaneous kinetic energy.
The equation (\ref{eq:operational_space_dyn}) can also be written in the form
\[
\bar{J}^{T} (B \vec{\ddot{q}} + C \vec{\dot{q}} + \vec{G}) = \vec{\gamma}
\]
resulting in 
\[
\vec{\gamma} = \bar{J}^{T} \vec{\tau}
\]
i.e. $\bar{J} = B^{-1} J^{T} \Lambda$ is a dynamically consistent generalized inverse of the Jacobian matrix
\end{frame}

\begin{frame}{Control of internal motions [Khatib, 1987]}
 In order to control the undesired internal motions a command torque $\vec{\tau}$ of the form
 \[
 \vec{\tau} = J^{T} \vec{\gamma} + (I_7 - J^{T}\bar{J}^{T}) \vec{\gamma}_{0}
 \]
 can be used where $\vec{\gamma}_{0}$ is projected in the null space of $\bar{J}^{T}$ hence not affecting
 the command wrench seen by the hand
 \par
 The final control law is
 \[
 \vec{\tau} = C \dot{\vec{q}} + \vec{G} + {}^{b}J^{T}_{S} ( B_A \vec{a} - B_A {}^{ws} \dot{J_{A,E}} \dot{\vec{q}} + {}^b\vec{w}_{S}) +
 (I_7 - ({}^{b}J^{T}_{S}) ({}^{b} \bar{J}^{T}_{S})) \vec{\gamma}_{0}
 \]
where
\[
\prescript{b}{}{\bar{J}}_{S} = B^{-1} ({}^{b} J^{T}_{S}) ({}^{b} \Lambda_{S}) \quad 
\prescript{b}{}{\Lambda}_{S} = ({}^{b} J_{S} B^{-1} ({}^{b} J^{T}_{S}))^{-1}
\]
\end{frame}

\begin{frame}{Design of the null operational wrench command $\vec{\gamma}_{0}$}
  To fix the orientation of the $5$-th link and
  to regulate the altitude between the $4$-th link and the hand 
  $\vec{\gamma}_0$ is chosen
  \[
  \vec{\gamma}_{0} = J_{im} ^{T} (K_p (\vec{x}_{im,des} - \vec{x}_{im} ) - K_d \vec{\dot{x}}_{im})
  \]
  \[
  \vec{x}_{im} =
  \begin{bmatrix}
    {}^{b} x_{l4} & {}^{b} y_{l4} & {}^{b} z_{l4} & \psi_{l5} & \theta_{l5} & \phi_{l5}
  \end{bmatrix}^{T} =
  \begin{bmatrix}
    \vec{p}^{T}_{l4} & \vec{\Phi}^{T}_{l5}
  \end{bmatrix}^{T}
  \]
  \[
  \begin{bmatrix}
  \vec{\dot{p}}_{l4} \\
  \vec{\dot{\Phi}}_{l5} \\
  \end{bmatrix} =
  J_{im} \vec{\dot{q}}
  \]
  \par
  \[
  K_p = \text{diag}(0, 0, k_{p,z}^{im}, 0, 0, k_{p, att}^{im})
  \quad
  K_d = k_{d}^{im} I_{6}
  \]
  \[
  \vec{p}_{l4} =
  \begin{bmatrix}
    * & * & {}^{b} p_{ee_{z}} + off_z
  \end{bmatrix}^{T}
  \quad
  \vec{\Phi}_{l5} =
  \begin{bmatrix}
    * & * & 0
  \end{bmatrix}^{T}
  \]
\end{frame}
